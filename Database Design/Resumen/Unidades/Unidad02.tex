\section{Unidad 2 - Teoría y Diseño de Bases de Datos Relacionales}

  \subsection{Formas Normales}
  Una Forma Normal es una restricción sobre un esquema de una base de datos relacional, contiene ciertas dependencias funcionales que
  previenen propiedades indeseables. Cumplir ciertas formas normales nos garantiza buenos diseños de bases de datos. El propósito de las
  formas normales es evitar redundancia, eliminar inconsistencias, anomalías de inserción y borrado.

  \subsubsection{Dependencias Funcionales}
  Una dependencia funcional es una restricción de valor único que se aplica sobre dos conjuntos de atributos X e Y de una relación R
  perteneciente a un esquema de base de datos. Constituye entonces una generalización del concepto de clave. Nos permite expresar hechos
  sobre el dominio que estamos modelando con nuestra base de datos. Esta restricción especifica que el valor del o los atributos en X
  determinan un único valor de los atributos en Y en todos los estados de la relación. Estás dependencias se utilizan para normalizar las
  distintas relaciones de un esquema de base de datos.

  \subsubsection{Primera Forma Normal - 1FN}
  Un esquema de relación R está en 1FN si los valores de los dominios del esquema de base de datos son atómicos (simples e indivisibles), es decir las valores de los
  atributos no son listas, cojuntos de valores o valores compuestos. 
  % La 1FN establece que los dominios de los atributos deben
  % incluir solo valores atómicos (simples, indivisibles) y prohíbe entonces las “relaciones dentro de relaciones” o las “relaciones como
  % atributos de tuplas”. \\ \\
  Ventajas de la 1FN
  \begin{itemize}
      \item Representación Tabular
      \item Lenguajes de Consulta más simples
      \item Definición de restricciones más simples
  \end{itemize}
  Desventajas de la 1FN

  \begin{itemize}
      \item Inconsistencia.
      \item Redundancia.
      \item Anomalías de inserción y de borrado.
  \end{itemize}
  Ejemplo de un esquema de relación en 1FN: \\
  \begin{table}[h]
    \begin{tabular}{| l | l | l | l | l| l|}
    \hline
    Id\_empleado & Dni & Nombre & Apellido & Id\_proyecto & Descripción\_proyecto\\ \hline
    1 & 45789777 & Martin & Perez & A1 & Proyecto sobre microprocesadores\\ \hline
    2 & 45789777 & Lucas & Martinez & A1 & Proyecto sobre programación\\ \hline
    3 & 33765922 & María & García & A2 & Proyecto sobre compiladores\\ \hline
    4 & 33765922 & Alejandra & Carrizo & A2 & Proyecto sobre compiladores\\ \hline
    5 & 33444222 & Berta & Gonzalez & A3 & Proyecto sobre compiladores\\ \hline
    \end{tabular}
  \end{table}

  \begin{itemize}
      \item Inconsistencia: En este ejemplo podemos ver inconsistencia en los datos ya que en la tupla con id\_empleado 1 y 2 podemos ver el
        mismo proyecto pero con distintas descripciones.
      \item Redundancia: Si vemos las tuplas  3 y 4 vemos que se repite información en este caso la descripción del proyecto.
      \item Anomalía de Inserción: Si insertamos una tupla con id\_proyecto A2 con descripción distinta a la de las tuplas ya existentes no
        podemos saber cual es la descripción correcta del proyecto A2.
      \item Anomalía de Borrado: Si borramos la tupla con id\_proyecto a3 perdemos la información tanto del proyecto como de la persona.
  \end{itemize}

  \subsection{Segunda Forma Normal - 2FN}
  Un esquema de relación R está en 2FN si está en 1FN y para cada atributo no primo en R tiene una dependencia funcional total con alguna
  clave de R. Es decir cada atributo no primo es irreduciblemente dependiente de una clave primaria. \\
  Como ejemplo veamos si el siguiente esquema está en 2FN: \\ \\
  V: Vuelo, D: Día, P: Piloto, G: Puerta \\

  \noindent \[
    R = \{D, G, P, V\} \hspace{1cm} F = \{VD \Rightarrow PG, V \Rightarrow G\} \hspace{1cm} K=\{VD\}
  \]

  \begin{itemize}
    \item Atributos primos: V, D.  
    \item Atributos no primos: G, P.  
  \end{itemize}

  \noindent Podemos ver que en la dependencia $V \Rightarrow G$ el atributo G es un atributo no primo, y depedende de V el cual es un
  atributo primo, es decir forma parte de la clave pero por si solo no es clave, entonces ya no se está cumpliendo la condición de que todo
  atributo no primo debe tener dependencia funcional total con una clave de R. Por lo tanto este esquema está en primera forma normal. \\

  \noindent Veamos si descomponiendo el esquema anterior podemos lograr que cumpla la 2FN: \\

  \noindent \[
    \rho = (R1, R2)=(VDP, VG) \hspace{1cm} F_{1} = \{VD \Rightarrow P\} \hspace{1cm} F_{2} = \{V \Rightarrow G\} \hspace{1cm} \\ \\
  \]

  Buscamos las claves utilizando el algoritmo siempre, nunca, tal vez:

  \begin{table}[h]
      \centering
      \begin{tabular}{l|l|l}
          Siempre & Nunca & Talvez \\
        \hline
          V & P & \\ 
          D & G & \\
      \end{tabular}
  \end{table}

  Las claves son: $K_{1}=\{VD\} \hspace{1cm} K_{2}=\{V\}$ \\

  Buscamos los atributos primos y no primos: 

  \begin{itemize}
    \item Primos: V, D
    \item No primos: P, G
  \end {itemize}

  Habiendo clasificado los atributos primos y no primos, encontrado las claves y usando las dependencias funcionales podemos observar que
  cada atributo no primo tiene una dependencia funcional total con la clave esto se cumple en cada esquema $(R_{1}, R_{2})$. Por lo tanto
  como cada esquema está en 2FN todo el esquema está en 2FN.

  \begin{table}[h]
      \centering
      \begin{tabular}{|l|l|l|l|}
        \hline
          Torneo & Año & Ganador & Fecha de nacimiento del ganador\\
        \hline
          Des Moines Masters & 1998 & Chip Masterson & 14 de marzo de 1977\\
        \hline
          Indiana Invitational & 1998 & Al Fredickson & 21 de julio de 1975\\
        \hline
          Cleveland Open & 1999 & Bob Albertson & 28 de septiembre de 1968\\
        \hline
          Des Moines Masters & 1999 & Al Fredrickson & 21 de julio de 1975\\
        \hline
          Indiana Invitational & 1999 & Chip Masterson & 14 de marzo de 1977\\
        \hline
      \end{tabular}
      \caption{Caption}
      \label{tab:my_label}
  \end{table}

  Esta relación está en 2FN pero presenta anomalía de actualización por ejemplo modificar la fecha de nacimiento provocaría que el mismo
  ganador tuviera dos fechas de nacimiento distintas.

  \subsection{Tercera Forma Normal - 3FN}
  \setlength{\parindent}{15pt} Un esquema de relación R está en 3FN con respecto a un conjunto de dependencias funcionales F si está en 2FN
  y cada atributo no primo no es transitivamente dependiente de una clave. \\
  Segunda definición: Un esquema de relación R está en 3FN con
  respecto a un conjunto de dependencias funcionales F si para cada dependencia funcional X -> A en F con A no perteneciente a X se verifica
  que: 

  \begin{itemize}
      \item X es superclave o
      \item A es primo
  \end{itemize}

  \noindent Como ejemplo veamos si el siguiente esquema está en 2FN:

  \begin{center}
    V: Vuelo, D: Día, I: IdPiloto, N: NombrePiloto
  \end{center}

  \[
    R = \{V, D, I, N\} \hspace{1cm} F = \{VD \Rightarrow IN, I \Rightarrow N, N \Rightarrow I\} \hspace{1cm} K=\{VD\}
  \] Buscamos los atributos primos y no primos: 

  \begin{itemize}
    \item Primos: V, D
    \item No primos: I, N
  \end{itemize}

  Observando las dependencias funcionales vemos que cada atributo tiene una dependencia funcional total con la clave de esquema, por lo
  tanto el esquema está en 2FN. Veamos si descomponiendo el esquema anterior podemos lograr que cumpla la 3FN: \\

  \noindent \[
    \rho = (R1, R2)=(VDI, IN) \hspace{1cm} F_{1} = \{VD \Rightarrow I\} \hspace{1cm} F_{2} = \{I \Rightarrow N, N \Rightarrow I\}
    \hspace{1cm} \\ \\
  \]

  \noindent Utilizando el algoritmo siempre, nunca, tal vez: \\

  \begin{itemize}
    \item Buscamos las claves en R1:
  \end{itemize}
  
  \begin{table}[h]
      \centering
      \begin{tabular}{c|c|c}
          Siempre & Nunca & Talvez \\
        \hline
          V & I & \\ 
          D &  & \\
      \end{tabular}
  \end{table}

  \begin{itemize}
    \item Buscamos las claves en R2:
  \end{itemize}

  \begin{table}[h]
      \centering
      \begin{tabular}{c|c|c}
          Siempre & Nunca & Talvez \\
        \hline
          I &  & \\ 
          N &  & \\
      \end{tabular}
  \end{table}

  \noindent Las claves son: $R_{1}: K=\{VD\} \hspace{0.5cm} R_{2}: K=\{I, N\}$ \\

  \noindent Buscamos los atributos primos y no primos: \\

  R1:

  \begin{itemize}
    \item Primos: V, D
    \item No primos: I
  \end {itemize}

  R2:

  \begin{itemize}
    \item Primos: I, N
    \item No primos: 
  \end {itemize}

  \noindent Conclusión: conociendo los atributos primos y no primos de ambas relaciones podemos observar que en ambos casos se cumple que el
  lado izquierdo de las dependencias funcionales es clave por lo tanto el esquema está en 3FN.
