\section{Unidad 3 - Modelo Relacional}

\subsection{Dependencias Funcionales}
Existe una teoría de diseño para las relaciones que nos permite examinar un diseño cuidadosamente y realizar mejoras basándonos un unos pocos principios.
La teoría comienza haciéndonos establecer las restricciones que se aplican a la relación. La restricción más común es la dependencia funcional es una declaración de un tipo que generaliza la idea de una clave para la relación.
En la teoria del diseño de bases de datos relacionales existe un concepto llamado dependencias funcionales el cual es una declaración de tipo que generaliza la idea de una clave para una relación.
Una Dependencia funcional sobre una relación R es una proposición de la forma:
Si dos tuplas de r coinciden en las atributos A1, A2, ... , An entonces también coincidirán en otro atributo B.
A1, A2, ... , An => B es decir decimos que A1, A2, ... An determinan funcionalmente a B
Es decir las tuplas poseen los mismos valores en sus componentes respectivas para cada uno de esos atributos entonces también coincidirán en otro atributo B.

\subsection{Restricciones de Integridad Referencial}
Las restricciones de integridad referencial se especifican entre relaciones y se utilizan para mantener la consistencia entre ellas. Para
lograr esto se utilizan las claves foráneas, con esto logramos que nuestra base de datos se mantenga en un estado consistente. Las
restricciones de integridad referencial determinan los estados de una base de datos, nos indican si está en un estado ilegal o no. Para
manejar peticiones ilegales a la base de datos planteamos estados ilegales para las claves foraneas con el objetivo de manejar dichos
estados.

\noindent Sobre la clave foránea se definen un conjunto de reglas que son:
\begin{itemize}
\item Regla de Nulos.
\item Regla de Borrado.
\item Regla de Actualización.
\end{itemize}
% \begin{itemize}
% \item Regla de Nulos: Establece si es posible colocar nulos en las claves foráneas.
% \item Regla de Borrado: Establece restricciones al momento de eliminar una tupla de la base de datos.
% \item Regla de cctualización: Establece restricciones al momento de modificar una tupla de la base de datos.
% \end{itemize}

\subsubsection{Regla de Nulos}
Establece si es posible colocar nulos en las claves foráneas.

\subsubsection{Regla de Borrado}
Si se intenta borrar una tupla cuya clave primaria es foranea en otra relación se puede aplicar una de las siguientes reglas:
\begin{itemize}
    \item Restringir: Esta regla restringe la acción de eliminar dicha tupla y no lo permite.
    \item Propagar: Esta regla permite borrar la tupla y luego borra las demás tuplas de la base de datos que tengan dicha clave como clave
      foranea.
    \item Colocar Nulos: Esta regla permite borrar la tupla y luego coloca el valor nulo en la clave foránea de todas las tuplas de la base
      de datos que tengan dicha clave como clave foránea.
\end{itemize}

\subsubsection{Regla de Actualización}
Si se intenta actualizar la clave primaria de una tupla y dicha clave es clave foranea en otra relación se puede aplicar una de las
siguientes reglas: 
\begin{itemize}
\item Restringir: Esta regla restringe la acción de modificar la clave primaria de dicha tupla y no lo permite.
\item Propagar: Esta regla permite modificar la clave primaria de la tupla y luego setea el nuevo valor en la clave foranea de las demás
  tuplas de la base de datos que tengan dicha clave como clave foranea.
\item Colocar nulos: Esta regla permite modificar la clave primaria de la tupla y luego coloca el valor nulo en la clave foránea de todas
  las tuplas de la base de datos que tengan dicha clave como clave foránea.
\end{itemize}

\subsection{Algebra Relacional}
El algebra relacional es el conjunto basico de operaciones para el modelo relacional. Estas operaciones permiten al usuario construir
consultas como expresiones de algebra relacional. El algebra relacional es muy importante por varias razones.
\begin{itemize}
    \item Provee los fundamentos formales para las operaciones del modelo relacional.
    \item Se utiliza como base para la implementación y optimización de consultas en el proceso de consulta y son una parte integral en los
      módulos de optimización en los sistemas gestores de bases de datos relacionales.
    \item Algunos de sus conceptos son incorporados dentro del lenguaje de consulta SQL standard.
\end{itemize}
\subsection{Cálculo Relacional de Tuplas - CRT}
El cálculo relacional provee un lenguaje declarativo de alto nivel para especificar consultas relacionales.
\subsection{Cálculo Relacional de Dominios - CRD}
\subsection{SQL Extendido}
% \vspace{2em}
