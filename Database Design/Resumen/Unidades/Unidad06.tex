\section{Mineria de Datos}
La minería de datos analiza datos almacenados en disco, si el volumen
de datos es muy grande se le da el nombre de Big Data las cuales son
grandes bases de datos. La minería de datos  extrae patrones o
información interesante (no trivial, implícita, previamente
desconocida y potencialmente util) de dichos datos.
Es importante marcar que realizar una consulta a la base de datos no
es lo mismo que realizar un trabajo de minería de datos. Al hacer una
consulta pedimos datos y una vez están disponibles los analizamos y
los utilizamos para las tareas que necesitemos. Por ejemplo: Si
necesitamos saber que ciudad registró mayor producción de manzanas
entre Cipolletti o Neuquén podemos realizar una consulta pidiendo los
datos de produccíon de manzana de las dos ciudades y comparar los
resultados y después puedo hacer suposiciones de por que hubo mayor
produccíon en una ciudad que en otra. En cambio con respecto a esa
pregunta las herramientas de minería van a tratar de determinar el
porque sucedio esto y no solo con los datos de la produccíon sino por
ejemplo con los datos climáticos, datos relacionados con el
mantenimiento de las plantas, etc. Entonces la minería no realiza
suposiciones sino que trata de descubrir relaciones y patrones
escondidos que no son siempre obvios.

% Página 1066 Fundamentals of database system
\subsection{Objetivos}
La minería de datos se lleva a cabo para alcanzar alguna meta o
utilizarla en ciertas aplicaciónes. En términos generales estos
objetivos entran en alguna de las siguiente clases:
\subsubsection{Predicción}
La minería de datos puede mostrar como ciertos atributos dentro de
los datos se comportaran en el futuro. Un ejemplo de minería de datos
predictiva puede ser analizar las transacciones de compras para
predecir que consumidores compraron si aplicamos ciertos descuentos,
cuanto volumen de venta generará una tienda en un cierto período y si
borrar una linea de productos dará más beneficios. En tales
aplicaciones la lógica de negocio se utiliza en conjunto con la
minería de datos. En un contexto científico ciertas ondas sísimicas
pueden predecir un terremoto con una alta probabilidad.
\subsubsection{Identificación}
Los patrones de datos pueden ser usados para identificar un item, un
evento o una actividad. Por ejemplo, intrusos intentando entrar en un
sistema informático pueden ser identificados por los programas
executados, archivos accedidos o tiempo de uso de la CPU por sesión.
\subsubsection{Clasificación}
La minería de datos puede particionar los datos en diferentes clases
o categorías basandose en una combinación de parámetros. Por ejemplo
los clientes de un supermercado pueden ser clasificados como
compradores de descuentos, rápidos, leales, leales a una marca o poco
frecuentes. Esta clasificación puede usarse en diferentes análisis de
transacciones de compra de clientes como actividad posterior a la
minería. 
\subsubsection{Optimización}
Un objetivo eventual de la minería de datos puede ser optimizar el
uso de recursos limitados como el tiempo, espacio, dinero o
materiales y con esto maximizar variables de salida como los
beneficios o ventas bajo un conjunto de restricciones. Este objetivo
se asemeja al función operación utilizada en problemas de
investigación que tratan con optimización bajo ciertas condiciones.

\subsection{Enfoques}
hola

% Doy ejemplos de tuplas con valores
\subsubsection{Aprendizaje Supervisado}
Los algoritmos de aprendizaje supervisado basan su aprendizaje en un
juego de datos de entrenamiento previamente etiquetados. Por
etiquetado entendemos que para cada ocurrencia del juego de datos de
entrenamiento conocemos el valor de su atributo objetivo. Esto le
permitirá al algoritmo poder “aprender” una función capaz de predecir
el atributo objetivo para un juego de datos nuevo.

Por ejemplo, una empresa de telecomunicaciones quiere entender por que
posee clientes leales y otros que no lo son. Entonces quiere predecir
que clientes no son leales y se iran con la competencia. Con esto en
mente el analista puede crear un modelo derivado de datos historicos
de sus clientes leales con aquellos que los han dejado dichos
datos seran etiquetados siguiendo un criterio como leales y no
leales. Una vez entrenado el modelo, utilizando un nuevo conjunto de
datos, será capas de predecir que clientes pueden cambiar de compañia,
es decir no son leales y cuales leales.

\subsubsection{Aprendizaje no Supervisado}
Los métodos no supervisados son algoritmos que basan su proceso de
entrenamiento en un conjunto de datos sin etiquetas o clases
previamente definidas. Es decir, a priori no se conoce ningún valor
objetivo o de clase, ya sea categórico o numérico. El aprendizaje no
supervisado está dedicado a las tareas de agrupamiento, también
llamadas clustering donde su objetivo es encontrar grupos similares
en el conjunto de datos. Por ejemplo, un grupo de vendedores quiere
conocer las similitudes que existen entre sus clientes para poder
crear y entender diferentes grupos de ventas y mercado. Puede haber
cliente sin hijos, clientes con ingresos bajos, clientes con más de
dos hijos, etc. Esto podría darnos por ejemplo 3 grupos:
\begin{itemize}
  \item Grupo 1, ingresos: Bajos, hijos: 1, auto: Lujoso.
  \item Grupo 2, ingresos: Bajos, hijos: 0, auto: Compacto.
  \item Grupo 3, ingresos: Medios, hijos: 2, auto: Sedán.
\end{itemize}

% Página 1089 Database Systems
\subsection{¿Por Qué Usar Minería de Datos?}
La minería de datos puede ser aplicada a una gran variedad de
contextos a la hora de la toma de decisiones en una organización.

\subsubsection{Marketing}
Se utiliza para analizar patrones en las compras de los consumidores.
Se pueden determinar varias estrategias de marketing como publicidad,
mails localizados, diseño de catalogos, distribución de sucursales,
etc. Por ejemplo vendedores por correo electronico pueden utilizar la
minería de datos para enviar mails a personas que puedan convertirse
en potenciales clientes en lugar de enviar correo a una lista de
millones de clientes.

\subsubsection{Análisis de Tendencias}
Entender las tendencias del mercado es util para reducir costos y
tiempo, en entidades financieras es bueno saber la forma en que los
clientes efectuan retiros y depósitos. Las cadenas de supermercado
utilizan minería de datos para analizar las tendencias del mercado y
ofrecer productos y promociones acorde a dichas tendencias.

\subsubsection{Detección de Fraudes}
Con técnicas de minería de datos podemos modelar cuales demandas de
seguro, llamadas telefónicas o compras con tarjeta de crédito es
probable que sean fraudulentas. Muchas organizaciones que
proporcionan tarjetas de crédito utilizan minería de datos para
modelar y entender los fraudes con tarjetas de crédito.

\subsection{El Proceso Para Minería de Datos}
\subsubsection{Preparación de los Datos}
Es el corazón de la minería de datos. Si los datos utilizados en un
estudio de minería de datos son de mala calidad los patrones que
obtenga de esos datos van a ser de mala calidad. Por lo tanto antes
de comenzar a entrenar el modelo debemos limpiar los datos es decir
verificar si cumplen con exactitud semántica, sintáctica,
completitud, si tengo valores nulos, etc. Además puede ser que haya
que normalizar datos para ajustarlos al estudio que se va a realizar
.
\subsubsection{Definición de un Estudio}
Todo proceso de minería de dato debe tener un objetivo o una
aplicación una vez encontrado el objetivo. Debemos elegir el método
de minería de datos utilizado es decir utilizar un aprendizaje
supervisado o no supervisado. Además no vamos a utilizar todo el
conjunto de datos para realizar el estudio lo que debemos hacer tomar
una muestra representativa para el estudio esto no es facil ya que
dicha muestra debe representar a todo el conjunto de datos.

\subsubsection{Lectura de Datos y Construcción del Modelo}
Una vez definido el estudio un software de minería de datos lee los
datos y construye un modelo. El modelo resume grandes cantidades de
datos en indicadores, algunos indicadores son:
\begin{itemize}
  \item Frecuencias: Muestran como ocurre cierto valor. Por ejemplo,
    el 40$\%$ de los clientes que dejan la compañia tienen un valor
    bajo en el indicador de estado.
  \item Pesos: Como algunas entradas son más válidas que otras.
    Podemos llegar a la conclusión de que el valor Bajo en el campo
    indicador\_de\_estado tiene un alto peso o impacto en los
    clientes perdidos.
  \item Conjunciones: Muchas veces algunas entradas tienen más peso
    juntas que separadas. Por ejemplo puede no ser cierto que el sexo
    másculino es más leal, pero si que las personas de sexo masculino
    que tienen mascotas y les guste esquiar si sean más leales.
  \item Diferenciación: Cuanto más es más importante una variable a
    otra. Por ejemplo: Quiero saber que jugadores de hockey son más
    probables que compren sus productos, pero si los jugadores están
    más predispuestos a comprar a su competidor lo más probable es
    que no se use este hecho.
  \item Ruido: Los modelos construidos muchas veces serán imperfectos
    debido a anomalías encontradas en los conjuntos de datos. Por
    ejemplo valores extraordinarios que se alejan de los valores
    normales. Existen filtros para disminuir el impacto del ruido en
    los conjuntos de datos y mejorar la correctitud del modelo.
    Muchos modelos possen topes que limitan la cantidad de ruido en
    ellos.
\end{itemize}

\subsection{Comprensión del Model}
\subsection{Mineria de Datos Aplicada a la Web}

\subsubsection{Web Content Mining}
Se trata de extraer del contenido de los documentos de la web. Se
puede clasificar en text minning para documentos planos, hypertext
mining si los documentos tienen enlaces a otros documentos o a si
mismo.
\subsubsection{Web Structure Mining}
Se intenta descubrir un modelo a partir de la topología de enlaces de
la red. Este modelo puede ser util para clasificar o agrupar
documentos.
\subsubsection{Web Usage Mining}
