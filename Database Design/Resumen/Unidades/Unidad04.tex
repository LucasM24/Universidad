\section{Unidad 04 - Modelo de Datos Objeto Relacional}
\noindent Existen tres enfoques para lograr persistencia de datos en el paradigma Orientado a Objetos.

\begin{enumerate}
  \item \textbf{Bases de Datos Orientadas a Objetos}: Usando un lenguaje orientado a objetos nativo.
  \item \textbf{Bases de Datos Objeto-Relacionales}: Agregando características orientadas a objetos a las bases de datos relacionales.
  \item \textbf{Mapeo Objeto-Relacional (ORM)}: Realizando conversiones entre ambos sistemas.
\end{enumerate}

\subsection{Bases de Datos Orientadas a Objetos}
Utilizamos lenguajes de programación orientados a objetos (Java, Smalltalk, etc) extendidos que nos permiten persistir información. La
información se representa mediante objetos como los presentes en la programación orientada a objetos. Es una buena elección cuando
necesitamos un buen rendimiento al manipular tipos de datos complejos. Proporcionan costos de desarrolo más bajos ya que al utilizar el
mismo tanto para programar como para persistir la información es solo cuestion de familiarizarse con funciones que permiten la persistencia.
Esto ayuda a agilizar el desarrollo y en cierta medida facilitar el mantenimiento tanto para la aplicación como para la base de datos.

Algunos de los conceptos de las bases de datos orientadas a objetos importantes:
\begin{itemize}
  \item Manejo de Objetos Complejos.
  \item Identidad de los Objetos.
  \item Lenguajes de Programación Orientados a Objetos.
\end{itemize}

\subsubsection{Objetos Complejos}
Un objeto complejo es un objeto que contiene en su definición otros objetos, esto forma una jerarquía de continentes. Por ejemplo: Un objeto
Bicicleta puede contener otros objetos como rueda, cambio, freno, etc. Un objeto complejo se representa mediante un grafo acíclico dirigido
(GAD).

\subsubsection{Identidad de los Objetos}
En orientación a objetos no es necesario que el usuario le proporcione un identificador al objeto, esto lo hace el sistema, al crearse un
neuvo objeto el sistema automaticaménte el asigna un identificador de objeto (OID). Estos identificadores son únicos e inmutables es decir
no cambian y no se reasignan. Al ser direcciones de memoria no son fáciles de recordar y además son únicos dentro de un sistema cuando yo
traduzco a otro sistema esos identificadores no se van a mantener.

\subsubsection{Lenguajes de Programación Orientados a Objetos}
Estos lenguajes permiten la persistencia de datos es decir almacenarlos en un disco duro, son lenguajes extendidos ya que mediante
constructores nos permiten el tratamiento de datos persistentes esto hace que el lenguaje de consulta esté totalmente integrado con el
lenguaje anfitrión.

Algunos inconvenientes de los lenguajes OO es que son potentes y complejos por lo que resulta más sencillo cometer errores que dañen la base
de datos. Es complejo realizar optimizaciones por ejemplo de reducción de E/S a disco ya que está integrado con el mismo lenguaje
posiblemente se podría realizar una optimización externa al lenguaje de programación.

\subsubsection{Mecanismos de Persistencia}
Los lenguajes de programación orientados a objetos realizan la persistencia utilizando mecanismos como constructores, funciones,
primitivas,etc. Hay cuatro tipos de formas de lograr persistencia:

\begin{itemize}
  \item Persistencia por Clases: Se declara una clase persistente y todos los objetos que sean instancias de dicha clase serán persistentes
    de manera predeterminada.
  \item Persistencia por Creación: Se crea la clase como transitoria es decir solo va a estar en memoria pero al momento de crear los
    objetos de esa clase de acuerdo a la manera de crear se decide si son objetos persistentes o no.
  \item Persistencia por Marca: Los objetos se crean originalmente como transitorios y al momento de decidir si se persiste o no, se marca
    efectivamente como persistente.
  \item Persistencia por Referencia: En este enfoque se declara como persistente uno o varios objetos (objeto raíz) de manera explícita y un
    objeto solo sera persistente si se hace referencia a él de manera directa o indirecta desde el objeto raíz. Esto provoca que sean
    persistentes estructuras de datos completas además puede ser costoso debido a la cantidad de referencias y se puede perder la pista de
    que objetos son persistentes y cuales no.
\end{itemize}

\subsubsection{Acceso a Objetos Persistentes}
Una vez que persistimos los objetos necesitamos encontrarlos en la base de datos al momento que necesitemos utilizarlos. Para encontrar
dichos objetos tenemos tres enfoques:
\begin{itemize}
  \item Nombrar Objetos: Podemos darle un nombre al objeto (como se hace con los archivos). Esto resulta poco práctico para grandes
    cantidades de objetos.
  \item Utilizar OID: Este enfoque no es muy util ya que los OID son punteros los cuales suelen ser muy largos y dificiles de recordar.
  \item Extensiones de Clases: Una extensión se define sobre una clase, posee una estructura de datos que contiene todos los objetos
    persistentes de dicha clase. Es decir la clase persona puede tener muchos objetos instanciados pero los que son persistentes se guardan
    en la extension de clase. Por ejemplo: La clase persona tendra muchos objetos persona creados algunos persistentes y otros no pero los
    objetos persitentes estaran guardados en la extensión de la clase persona. Una clase con extensión puede tener una o más claves. Esta
    clave puede tener una o más propiedades (atributos y asociaciones) cuyos valores están restringidos a ser unicos por cada objeto en la
    extensión.
\end{itemize}

\subsubsection{Comparación BDOO - BDR}
Relaciones:

\begin{itemize}
  \item DBO: Las relaciones se realizan por medio de referencias a otro objetos pueden ser objetos de la misma clase o no, utilizando los
    OID. Estas relaciones pueden ser unidireccionales o bidireccionales (mediante inversas generando integridad referencial). Solo existen
    relaciones binarias.
  \item DBR: Por medio de valores de atributos (claves foráneas). Pueden existir relaciones N-arias.
\end{itemize}

Claves:

\begin{itemize}
  \item DBO: No son necesarias existen los OID. A pesar de que existen los OID son direcciones de memoria lo cual las hace dificil de
    recordard por eso se utilizan las claves en las extensiones de clases para poder identificar de una forma más sencilla a los distintos
    objetos de una clase.
  \item DBR: Las claves son la base del modelo relacional. Tenemos las claves primarias y las claves foráneas utilizadas para implementar
    integridad referencial. 
\end{itemize}

Herencia:

\begin{itemize}
  \item DBO: Se heredan todas las propiedades (atributos, relaciones y operaciones). No puede ser solapada.
  \item DBR: Se utiliza la restricción de clave primaria y foránea.
\end{itemize}

Operaciones:

\begin{itemize}
  \item DBO: Se especifican durante el diseño de la base de datos.
  \item DBR: Se especifican en etapas siguientes.
\end{itemize}

\noindent \textbf{Lenguajes de Consulta}:

A pesar de que los mecanismos de persistencia están integrados dentro del mismo lenguaje existen dos lenguajes de consulta para BDOO que
son.

\begin{itemize}
  \item DBO:Object Definition Language - Lenguaje de Definición de Objetos (ODL) y Object Query Language - Lenguaje de Consulta de Objetos
    (OQL). 
  \item DBR: Se propone SQL con Data Definition Language - Lenguaje de Definición de Datos (DDL) y Data Manipulation Language - Lenguaje de
    Manipulación de Datos (DML).
\end{itemize}

% TODO Copiar la tabla de correspondencia en MER y UML setear encabeza en el centro y contenido alineado a la izquierda
\subsubsection{Comparación MER - UML}
\begin{table}[h]
    \centering
    \begin{tabular}{|c|c|} % Bordes en todas las celdas
        \hline
        MER & UML \\ 
        \hline
        Entidad & Clase \\ 
        \hline
        Atributo & Atributo \\ 
        \hline
        Relación Binaria    & Asociación \\ 
        \hline
        Relación con atributos & Clase asociación \\ 
        \hline
        Generalización/Especialización & Superclase/SubclaseDato \\ 
        \hline
        Relación N-aria & Clase separada \\ 
        \hline
    \end{tabular}
\end{table}

\subsection{Bases de Datos Objeto-Relacionales}
Al no tener las bases de datos orientadas a objetos demasiada demanda, surgieron como alternativa las bases de datos objeto-relacionales a
estas bases de datos se le agregan algunas características más interesantes y más complejas como:
\begin{itemize}
  \item \textbf{Constructores de tipos}
  \item \textbf{Identidad de Objetos}: A través del uso de referncias de tipos.
  \item \textbf{Herencia}: A traves de la palabra UNDER/OF.
  \item \textbf{Objetos Complejos}: Los dominios de los atributos pueden ser atomicos o de relación.
\end{itemize}

\subsubsection{Constructores de Tipos}
Para especificar estructuras de objetos complejos se utilizan los user-defined types (UDTs) y a su vez poder separar la declaracion de tipos
de la creación de una tabla. En cuanto a los tipos acepta tipos primitivos y además utilizar atributos que sean estructuras de datos como
list, set, setlist, etc. Al permitir que un atributo sea una estrucutra de datos ya perdemos la propiedad de atomicidad en dichos atributos
ya que pueden tener más de un valor. \\

\begin{lstlisting}[language=OQL]
CREATE TYPE tipo_telefonos AS(
  notipo VARCHAR(5),
  codigo_area INTEGER,
  numero VARCHAR(10) 
);
\end{lstlisting}

\begin{lstlisting}[language=OQL]
CREATE TYPE tipo_persona AS(
  dni VARCHAR(20),
  nombre VARCHAR(20),
  apellido VARCHAR(20),
  fecha_nacimiento DATE,
  telefonos tipo_telefonos ARRAY[4],
)
\end{lstlisting}

\subsubsection{Identidad de Objetos}
\noindent Los OID pueden crearse por medio de tipos de referencia se utilizan para tipos o tablas luego de la creación (de tipos o tablas). \\

\begin{lstlisting}[language=OQL]
REF IS <atributo OID> <metodo de generación>
\end{lstlisting}

\noindent La siguiente sentencia indica que si una nueva tipo\_persona es creada el sistema le asignara un OID.

\begin{lstlisting}[language=OQL]
CREATE TABLE Persona OF tipo_persona REF IS id_persona SYSTEM GENERATED
\end{lstlisting}

\noindent La siguiente sentencia indica que si una nueva tipo\_persona es creada se utilizar el sistema tradicional de clave primaria
provista por el usuario.

\begin{lstlisting}[language=OQL]
CREATE TABLE Persona OF tipo_persona(
  REF IS id_persona DERIVED;
  PRIMARY KEY (dni);
)
\end{lstlisting}

\subsubsection{Herencia}
En los paradigmas orientados a objetos existe la herencia, en este caso tenemos dos tipos de herencia, herencia de tipos que permite heredar
los atributos de un tipo definido. Los tipos que heredan los atributos se definen con la palabra \textbf{under} y son considerados subtipos
mientras que el tipo del cual heredan los atributos se lo considera supertipo.

\begin{lstlisting}[language=OQL]
CREATE TYPE tipo_persona AS(
  dni VARCHAR(20),
  nombre VARCHAR(20),
  apellido VARCHAR(20),
  fecha_nacimiento DATE,
  telefonos tipo_telefonos ARRAY[4],
)
\end{lstlisting}

\begin{lstlisting}[language=OQL]
CREATE TYPE estudiante UNDER tipo_persona(
  curso VARCHAR(20),
  dpto VARCHAR(20)
)
\end{lstlisting}

\begin{lstlisting}[language=OQL]
CREATE TYPE profesor UNDER tipo_persona(
  sueldo INTEGER
)
\end{lstlisting}

Por otro lado tenemos la herencia de tablas (OF) que se corresponde con la generalización/especialización utilizado en el modelo relacional.
Para ilustrar esto concepto usamos el siguiente ejemplo creamos una tabla gente de tipo tipo\_persona y creamos la tabla estudiante de tipo
estudiante y la tabla profesor de tipo profesor. Las tablas estudiante y profesor las definimos como subtablas de la tabla Gente. Esto
herencia la momento de insertar una tupla en estudiante o en profesor dicha tupla también estara presente en la tabla gente. Funciona como
una especialización total y dijunta.

\subsubsection{Objetos Complejos}
Se soportan dos tipos nuevos arrays y multiset ambos pueden ser de cualquier tipo VARCHAR, INTEGER, etc o del tipo de otra tabla.
\begin{itemize}
  \item \textbf{Array}: Es una colección ordenada de elementos del mismo tipo. Se accede como array[i].
  \item \textbf{Multiset}: Es una colección no ordenada de elementos del mismo tipo que pueden ocurrir muchas veces.
  \item \textbf{Referencia}: Hace referencia a otra tabla. Puede ser una referencia a un array o multiset que contenga referencias a otras
    tablas existentes (relaciones anidadas).
\end{itemize}

Al tener relaciones que poseen atributos que son arreglos o conjuntos ya no soportan 1FN ya que se pierde la propiedad de atomicidad. Esto
no quiere decir que todas las aplicaciones se modelan mejor mediante relaciones en 1FN. Por ejemplo los atributos multievaluados es mejor
modelarlos utilizando objetos complejos ya que evitamos crear tablas adicionales y por lo tanto hacer joins adicionales. Las relaciones
anidadas se navegan entre clases de una forma distinta a los joins.

Para realizar consultas sobre los MOR (Modelo Objeto-relacional) se utilizan nuevos operadores que son unnest y collect. Con unnest podemos
acceder a los elementos de las colecciones y/o listarlos de forma tabular. El collect es la inversa del unnest donde se puede transformar
aalgo tabular en 1FN en una colección.

% TODO copiar ejemplas de las transparencias, ejemplos para acceder a colecciones de referencias, etc.
\subsection{Mapeo Objeto-Relacional - ORM}
Es el tercer enfoque para integrar POO con bases de datos. Es una capa que se construye sobre el modelo relacional, el programador define un
mapeo entre las tuplas del modelo relacional y los objetos de un lenguaje OO con la diferencia de que aquí los objetos son transitorios y no
hay una identidad de objeto permanente. Esti permite a los programadores construir aplicaciones usando el modelo orientado a objetos
mientras que se utiliza una base de datos relacional para almacenar los datos. Con esto la interacción con la base de datos se realiza
mediante un lenguaje de programación.

Ventajas:
Se interactua directamente con los objetos.
Reduce el tiempo de código.
Se puede trabajar en memoria.
Se puede cambiar el SGBD de forma sencilla ya que un orm te abstrae del gestor que se este utilizando para las consultas.
Se puede trabajar en memoria.

Desventajas
Toma tiempo aprender a utilizarlo y posteriormente utilizarlo de forma eficiente.
Se pierde el control de como se trabaja por debajo.
Los cambios son dificiles de realizar sobre todo una vez que esta funcionando.
