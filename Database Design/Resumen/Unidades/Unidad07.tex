\section{Unidad 07 - Bases de Datos Distribuidas}
Consiste en un conjunto de nodos que están conectados entre si por
medio de una red y cada nodo hay un sistema gestor de bases de datos
que puede ser de distinta arquitectura pero todos esos sitios donde
hay un sistema gestor están de acuerdo de trabajar juntos. Es por que
hay un monton de coordinaciones que se deben hacer entre los sitios
para ejecutar las transacciones en una base de datos ya que requieren
que trabajen juntos.

Es una colección de multiples bases de datos interrelacionadas
lógicamente distribuidas a través de una red de comunicación, y un
sistema de administración de bases de datos distribuidas (DDBMS) como
un sistema de software que administra una base de datos distribuidas
mientras hace que la distribución sea transparente para el usuario.
Con esto un usuario en un sitio puede acceder a datos en cualquier
lugar de la red como si esos datos estuvieron almacenados en su
propio sitio.

\subsection{Características de una Base de Datos Distribuida}
Para que una base de datos se llame distribuida deben cumplirse las
siguientes condiciones mínimas.

\begin{itemize}
  \item Es fundamental que las bases de datos estén lógicamente
    relacionadas.
  \item Debe existir conexión entre los nodos a traves de una red.
  \item No es necesario que todos los nodos sean identicos en
    términos de datos, hardware y software.
\end{itemize}

\subsubsection{Ventajas}
\begin{itemize}
  \item Naturaleza distribuida de algunas aplicaciones: Combina con
    el diseño de las empresas es decir, estas se dividen por sectores,
    departamentos entonces los datos del sector serían locales para
    cada sector y los demás se acceden cuando se los necesite.
  \item El diseño distribuido combina, la eficencia de procedimiento
    es decir los datos están almacenados cerca del punto del cual se
    utilizan con mayor frecuencia y provee un mayor accesiblidad ya
    que se puede acceder a datos de otros nodo  a través de la red de
    comunicaciones.
  \item Posibilidad de compartir los datos y a su vez ejercer un
    control sobre los datos locales.
  \item Mayor rendimiento: Bases de datos locales más pequeñas, menor
    carga de transacciones y el procesamiento de varias transacciones
    a distintos sitios puede ejecutarse en paralelo.
\end{itemize}
\subsubsection{Desventajas}
\begin{itemize}
  \item Aumento de complejidad tanto en el diseño como la
    implementación del sistema distribuido base de datos.
  \item Debe tener capacidad de acceder a sitios remotos, elaborar
    estrategias para la ejecución de consultas y transacciones que
    accedan a más de un sitio, determinar a que sitio se va a buscar
    la información, mantener la consistencia entre los sitios y
    recuperar a caídas en sitios individuales.
\end{itemize}

\subsection{Características}
Para que una base de datos se llame distribuida deben cumplirse las
siguientes condiciones mínimas.
\subsubsection{Condición Primordial}
Es la base de todas las demás características, se basa en que para un
usuario, un sistema distribuido debería verse exactamente igual que
un sistema no distribuido y todos los problemas de un sistema
distribuido deberían ser a nivel de implementación o internos.
\subsubsection{No Debe Haber Dependencia de un Sitio Central}
No debe haber dependencia de un sitio central para obtener un
servicio maestro por ejemplo, por ejemplo un procesamiento central
para las consultas o una administración centralizada para las
transacciones, de modo que todo el sistema dependa de ese sitio
central. Esto es por que el sitio central puede ser un cuello de
botella o también todo el sistema se vuelve más vulnerable ya que si
el sitio sufriera un despefecto todo el sistema dejaría de funcionar.
\subsubsection{Operación Continua}
El sistema nunca debe apagarse a proposito para realizar operaciones
como añadir un nuevo sitio o instalar una versión mejorada del SGBD
en un sitio ya existente.
\subsubsection{Autonomía}
La autonamía determina hasta que punto los nodos existentes o las
bases de datos en un sistema de bases de datos conectadas pueden
operar de forma independiente. Se definen varios tipos:
\begin{itemize}
  \item Autonomía de diseño: Se refiere a la indenpendencia del uso
    del modelo de datos y las técnicas de gestión de transacciones
    entre los nodos. 
  \item Autonomía de Comunicación: Determina hasta que punto cada
    nodo puede compartir información con otros nodos.
  \item Autonomía de Ejecución: Se refiere a la independencia de los
    usuario para actuar como deseen.
\end{itemize}
