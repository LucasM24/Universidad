\section{Calidad de Datos}
Hemos visto varios mecanismos para asegurar calidad de los datos en nuestra base de datos, tales como brindar una semantica precisa definir
dependencias funcionales para mantener integridad referencial, distintas formas normales que nos ayudan a tratar redundancia y anomalía de
datos manejo de valores nulos y evitar la generación de datos falsos evitando perder dependencias al momento de realizar una descomposición.
Pero en este caso nos referimos a la \textbf{Calidad de los datos} como la calidad de los valores de los datos, es decir que tan util es un
dato para nuestro sistema o dominio. El concepto es relativo por que un dato que necesita ser utilizado de una determinada manera en un
dominio y ese dato no necesita ser utilizado de la misma manera en otro dominio.
Es como analizar el data en cuanto a como y para que se va a utilizar el dato de forma tal que uno diga bueno si el dato es usado y usado de
tal manera así tiene que ser guardado ese dato. Entonces yo me garantizo que ese dato como conozco como va a ser utilizado voy a hacer todo
lo posible para que este almacenado de esa manera.
Es relativo por que un dato que necesita ser de una manera en una base de datos no necesita ser usado de la misma forma en otra base de
datos. Por ejemplo direccion un dominio solo basta con guardad el nombre de la ciudad por ejemplo Cipolletti y en otro domino debemos
guardar más detalles Río Neuquén 62, Cipolletti.

Definición: La calidad de los datos es la capacidad de los datos de cumplir con su propósito en un determinado contexto. Se puede resumir
como \textbf{capacidad de uso}.

\subsection{Reglas Basadas en el Uso de Datos}
\begin{enumerate}
  \item Los datos que no son usados no se mantienen correctos por mucho tiempo. Por ejemplo en un sistema para administrar pago de servicios
    el email es un dato muy importante ya que por este medio se envía la factura de dicho servicio por lo tanto al tener un uso frecuente
    ese dato se mantiene correcto. En cambio un sistema que solo utilice el mail en casos particular puede ser que al momento de utilizarlo
    ese dato esté desactualizado.
  \item La calidad de los datos de un sistema está en función de su uso, no de su obtención. Esto quiere decir que podemos obtener mucha
    información de una forma eficiente y rápida pero esa información puede no ser utilizada por completo o directamente no ser utilizada por
    lo cual dicha información pierde valor.
    \item La calidad de los datos no será mejor que su uso más riguroso. La calidad del dato está dada por el uso del dato, si el dato es
      usuado muy frecuentemente va a tener una calidad alta.
    \item La calidad del datos tiende a volverse peor con el paso del tiempo.
    \item La baja probabilidad de que un dato cambie repercutira en su calidad. Si no tenemos mecanismos que nos permitan mantener
      actualizado dicho dato su calidad ira disminuyendo.
    \item Las reglas de calidad de datos se aplican tanto a los datos como a los metadatos (datos sobre los datos).
\end{enumerate}

\subsection{Programa de Calidad Basado en el Uso}
Cuando tenemos una base de datos funcionando podemos aplicar este programa para mejorar la calidad de los datos de nuestra base de datos.

\begin{enumerate}
  \item \textbf{Auditar}: Debo preguntarme como de buenas son los datos de la BD hoy. Algunas de estas preguntas pueden ser que datos nos
    interesan, quien usa los datos, con que proposito son utilizados, se usan actualmente, que tan actuales son, que tan utiles son.
  \item \textbf{Rediseñar}: Identifico areas de datos críticas, consultar a usuarios si utilizan dichos datos críticos. Los datos se usan no
    se usan, verificar el por que de la existencia en caso de haber datos nulos, por que son nulos. Compartir los datos con más usuarios
    siempre teniendo presente la seguridad del sistema. Desarrollar metadatos para datos nulos o con poca calidad.
  \item \textbf{Entrenar}: Comunicar a las personas que utilizan la base de datos la importancia de la calidad de datos y llevar a cabo
    planes de entrenamiento y educación para mejorar la calidad de dichos datos y deben preguntarse para que se van a utilizar esos datos
    para así poder entender lo importancia de que un dato se cargue correctamente.
  \item \textbf{Medir}: Medir constantemente la calidad de los datos por medio de técnicas automatizadas y verificar la veracidad de los
    datos es decir verificar si tienen sentido en la realidad o dicho de otra forma auditar fisicamente. Esto debe hacerse con cierta
    recurrencia para mantener la base de datos en un estado optimo la mayor parte del tiempo posible.
\end{enumerate}

Fomentar el Dato
Esto es muy importante para la calidad de los datos ya que utilizar los datos que tenemos almacenados en la DB es la mejor forma de
mantenerlos con calidad esto además nos ayuda a identificar quienes son los que usan esos datos, que uso se les da a los datos y por cada
dato podemos evaluar su calidad a partir de distintas dimensiones.

\subsection{Dimensiones de Calidad}
Son atributos de calidad deseables para los datos, cada dimensión captura un aspecto especifico incluido dentro de los requerimientos de
calidad de datos. La calidad de datos puede ser vista como multidimensional es decir definido mediante las distintas dimensiones.

\begin{itemize}
  \item Exactitud: Mide el nivel de correspondencia entre el dato y la vida real. Es decir si los datos representan exactamente la realidad.
    Por ejemplo: Un dato de inventario nos indica que en el deposito hay tres celulares, en el momento que vayamos al deposito y revisemos
    debemos encontrar los tres celulares que indica el dato.
  \item Exactitud Sintactica: Es la cercanía del valor del dato con los elementos del dominio en el cual está definido.
  \item Exactitud Semántica: Es la cercanía del valor del dato con su valor real. Por ejemplo si tenemos la pelicula Casablanca podemos
    tener un atributo director con valor Martinez es sintacticamente correcto pero no es Martinez el director de la película Casablanca.
  \item Completitud: Es un aspecto en el cual debemos saber que cantidad de información es necesaria en el dato para considerarla completa.
    Es decir poder responder a la pregunta Todos los datos necesarios están presentes? Por ejemplo: Para un dominio la dirección puede ser
    considerarse un dato completo que solo tenga el nombre de la ciudad donde vive una persona pero la dirección de un dominio de ventas es
    probable que el atributo direccion necesite de más datos es decir ciudad, provincia, altura, cp, etc. La completitud depende mucho del
    contexto en el cual se va a utilizar un dato. Si hablamos de completitud en el caso de los nulos también es una cuestion que tiene que
    ver con el contexto por ejemplo: En un tabla que almacena personas para un atributo dirección si tiene el valor null no sabemos si fue
    un error o si esa persona no tiene un lugar donde vivir, en cambio si en un atributo email tenemos el valor null, podemos considerarlo
    completo ya que puede pasar que una persona no tenga correo electrónico.
  \item Consistencia: Este aspecto se refiere a consistencia y comprensión de los datos. Por ejemplo: Si queremos guardar el género de una
    persona y en una tupla el campo dice hombre, pero en otro dice M se puede confundir con M de masculino o M de mujer. Por eso se necesita
    consistencia al cargar datos en el sistema para esto se pueden definir protocolos que deban cumplir las personas encargadas de cargar
    datos. Puede haber casos con multiples atributos como puede ser el caso de las fechas por ejemplo un error seria tener la fecha de
    nacimiento de una persona sea menor a la fecha de ingreso a un trabajo.
  \item Actualidad: Con que frecuencia se actualizan los datos. Por ejemplo la dirección de una persona.
  \item Volatibilidad: Frecuencia con la que el dato se mantiene válido. Por ejemplo: fecha de nacimiento vs stock de productos.
  \item Temporalidad: Cuan actuales son los datos para realizar una tarea. Pueden ser utiles pero no estar disponibles cuando se necesitan.
    Por ejemplo: Si publican el 21 de febrero que la fecha del parcial es el 20 de febrero.  En este caso el dato es util pero no está
    disponible cuando se necesita.
\end{itemize}
