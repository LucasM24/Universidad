\section{Unidad 05 - Calidad de Datos}
Hemos visto varios mecanismos para asegurar calidad de los datos en nuestra base de datos, tales como brindar una semantica precisa definir
dependencias funcionales para mantener integridad referencial, distintas formas normales que nos ayudan a tratar redundancia y anomalía de
datos manejo de valores nulos y evitar la generación de datos falsos evitando perder dependencias al momento de realizar una descomposición.
Pero en este caso nos referimos a la \textbf{Calidad de los datos} como la calidad de los valores de los datos, es decir que tan util es un
dato para nuestro sistema o dominio. El concepto es relativo por que un dato que necesita ser utilizado de una determinada manera en un
dominio y ese dato no necesita ser utilizado de la misma manera en otro dominio.
Es como analizar el data en cuanto a como y para que se va a utilizar el dato de forma tal que uno diga bueno si el dato es usado y usado de
tal manera así tiene que ser guardado ese dato. Entonces yo me garantizo que ese dato como conozco como va a ser utilizado voy a hacer todo
lo posible para que este almacenado de esa manera.
Es relativo por que un dato que necesita ser de una manera en una base de datos no necesita ser usado de la misma forma en otra base de
datos. Por ejemplo direccion un dominio solo basta con guardad el nombre de la ciudad por ejemplo Cipolletti y en otro domino debemos
guardar más detalles Río Neuquén 62, Cipolletti.

Definición: La calidad de los datos es la capacidad de los datos de cumplir con su propósito en un determinado contexto. Se puede resumir
como \textbf{capacidad de uso}.

\subsection{Reglas Basadas en el Uso de Datos}
\begin{enumerate}
  \item Los datos que no son usados no se mantienen correctos por mucho tiempo. Por ejemplo en un sistema para administrar pago de servicios
    el email es un dato muy importante ya que por este medio se envía la factura de dicho servicio por lo tanto al tener un uso frecuente
    ese dato se mantiene correcto. En cambio un sistema que solo utilice el mail en casos particular puede ser que al momento de utilizarlo
    ese dato esté desactualizado.
  \item La calidad de los datos de un sistema está en función de su uso, no de su obtención. Esto quiere decir que podemos obtener mucha
    información de una forma eficiente y rápida pero esa información puede no ser utilizada por completo o directamente no ser utilizada por
    lo cual dicha información pierde valor.
    \item La calidad de los datos no será mejor que su uso más riguroso. La calidad del dato está dada por el uso del dato, si el dato es
      usuado muy frecuentemente va a tener una calidad alta.
    \item La calidad del datos tiende a volverse peor con el paso del tiempo.
    \item La baja probabilidad de que un dato cambie repercutira en su calidad. Si no tenemos mecanismos que nos permitan mantener
      actualizado dicho dato su calidad ira disminuyendo.
    \item Las reglas de calidad de datos se aplican tanto a los datos como a los metadatos (datos sobre los datos).
\end{enumerate}

\subsection{Programa de Calidad Basado en el Uso}
Cuando tenemos una base de datos funcionando podemos aplicar este programa para mejorar la calidad de los datos de nuestra base de datos.

\begin{enumerate}
  \item \textbf{Auditar}: Se debe revisar los datos que hay en la BD
    actualmente y comenzar a hacer preguntas sobre los datos. ¿Qué
    datos nos interesan?, ¿Quien usa los datos?, ¿Para que se
    utilizan?, ¿Se usan actualmente?, ¿Qué tan actuales son?, etc. A
    partir de preguntas como estas podemos auditar la calidad de los
    datos que disponemos actualmente.
  \item \textbf{Rediseñar}: Identificar áreas de datos criticos en el
    sistema, verificar si existen nulos entre esos datos, consultar a
    los usuarios si dichos datos son útiles o están siendo
    utilizados. A partir de esto podemos rediseñar dichas areas para
    que más usuarios puedan utilizar dichos datos críticos con el fín
    de mantener su calidad y lograr que sean utilizados
    frecuentemente, averiguar la razón de los datos nulos y prevenir
    que se sigan generando buscando la raíz del problema y planteando
    una solución, desarrollar metadatos para datos nulos o datos con
    poca calidad. Todo esto siempre teniendo presente la seguridad de
    los datos y del sistema.
  \item \textbf{Entrenar}: Comunicar a las personas que utilizan la
    base de datos, la importancia de la calidad de datos y llevar a
    cabo planes de entrenamiento y educación para mejorarla.
    Plantear a los usuario la pregunta ¿para qué se van a utilizar
    los datos? ayudará a comprender la importancia de cargar un dato
    correctamente.
  \item \textbf{Medir}: Medir constantemente la calidad de los datos
    por medio de técnicas automatizadas y verificar su exactitud
    semántica. Esto debe realizarse con cierta recurrencia para
    mantener la base de datos en un estado optimo la mayor parte del
    tiempo posible.
\end{enumerate}

Fomentar el Dato
Esto es muy importante para la calidad de los datos ya que utilizar los datos que tenemos almacenados en la DB es la mejor forma de
mantenerlos con calidad esto además nos ayuda a identificar quienes son los que usan esos datos, que uso se les da a los datos y por cada
dato podemos evaluar su calidad a partir de distintas dimensiones.

\subsection{Dimensiones de Calidad}
Son atributos de calidad deseables para los datos, cada dimensión captura un aspecto especifico incluido dentro de los requerimientos de
calidad de datos. La calidad de datos puede ser vista como multidimensional es decir definido mediante las distintas dimensiones.

\begin{itemize}
  \item Exactitud Semántica: Mide el nivel de correspondencia entre
    el dato y la vida real. Es decir si los datos representan
    exactamente la realidad. Por ejemplo: Un dato de inventario nos
    indica que en el deposito hay tres celulares, en el momento que
    vayamos al deposito y revisemos debemos encontrar los tres
    celulares que indica el dato.
  \item Exactitud Sintáctica: Es la cercanía del valor del dato con
    los elementos del dominio en el cual está definido.
  \item Completitud: Se refiere a que cantidad de cantidad de
    información es necesaria en el dato para considerarla completa.
    Poder responder a la pregunta ¿todos los datos necesarios están
    presentes? Por ejemplo: Para un dominio la dirección puede
    considerarse un dato completo utilizando solamente el nombre de
    la ciudad donde vive una persona pero la dirección de un dominio
    orientado a ventas a domicilio es probable que el atributo
    direccion necesite más datos ya que necesita ser más preciso, por
    ejemplo incluir ciudad, provincia, altura, cp, etc. En el caso de
    los nulos podemos plantear el siguiente ejemplo: en un tabla que
    almacena personas para un atributo dirección si tiene el valor
    null no sabemos si, fue un error, si esa persona no tiene un
    lugar donde vivir, en cambio si en un atributo email tiene el
    valor null podemos considerarlo completo ya que puede pasar que
    una persona no tenga correo electrónico. La completitud está
    ligada al contexto en el cual se va a utilizar un dato. 
  \item Consistencia: Este aspecto se refiere a la igualdad entre
    valores del mismo dato y a su vez esto afecta a la comprensión de
    dicho dato. Por ejemplo: Si queremos guardar el género de una
    persona y en una tupla el campo tiene el valor hombre, en otra
    tupla tiene el valor M esto trae problemas ya que no se sabe si
    M se refiere a masculino o a mujer. Por esto la consistencia es
    importanta al momento de cargar datos en el sistema. Una forma de
    prevenir inconsistencia es definir protocolos que deban cumplir
    las personas encargadas de cargar datos. Otro caso de
    inconsistencia puede ser con multiples atributos utilizando
    fechas, un error sería tener que la fecha de nacimiento de una
    persona sea menor a la fecha de ingreso a un trabajo.
  \item Actualidad: Con que frecuencia se actualizan los datos. Por
    ejemplo la dirección de una persona.
  \item Volatibilidad: Frecuencia con la que el dato se mantiene
    válido. Por ejemplo: fecha de nacimiento vs stock de productos.
  \item Temporalidad: Cuan actuales son los datos para realizar una
    tarea. Pueden ser utiles pero no estar disponibles cuando se
    necesitan. Por ejemplo: Si publican el 21 de febrero que la fecha
    del parcial es el 20 de febrero.  En este caso el dato es util
    pero no está disponible cuando se necesita.
\end{itemize}

\subsection{La Base de Datos Como un Lago}
Veamos la analogía de una base de datos como una lago podemos establecer las siguientes correspondencias:
\begin{itemize}
  \item Agua del lago = Datos.
  \item Ríos = Cadenas de información que crean los datos.
  \item Fábricas = Fuentes de contaminación.
  \item Personas tomando agua = Consumidor de los datos.
\end{itemize}

\noindent Ante la contaminación del lago podemos:
\begin{enumerate}
  \item No hacer nada y tratar a las personas afectadas por tomar el agua del lago.
  \item Filtrar el agua del lago, eliminar las fuentes de contaminación y poner el agua en el lago.
  \item Filtrar pequeñas cantidades de agua cada día. Por ejemplo filtrar el agua de los ríos antes de que entre el agua al lago o filtrar el
    agua antes de su consumo.
  \item Identificar las fuentes de contaminación y eliminarlas (o al menos mitigarlas).
\end{enumerate}

\noindent Del mismo modo, frente a una base de datos contaminada (pobre calidad de datos) tenemos las mismas cuatro opciones:
\begin{enumerate}
  \item Tratar con los impactos  que generan los datos erroneos, lo cual puede derivar en tomar malas decisiones corregir y/o calmar
    clientes enojados. Por ejemplo: Devolver el dinero a un cliente.
  \item Realizar una limpieza general de toda la base de datos, lo cual es bastante costoso. Comparar con otras bases de datos para buscar
    diferencias y ejecutar pruebas automatizadas para identificar datos que no se ajustan a las reglas de negocio.
  \item Realizar pequeñas limpiezas de la base de datos periodicamente. Podemos limpiar los datos a la entrada de una base de datos o antes
    de usarlos.
  \item Encontrar las fuentes erroneas en los procesos de la organización que hacen que nuestros datos se contaminen.
\end{enumerate}

Enfrentarse a la realidad:
Podemos ver las opciones anteriores como distintos programas para enfrentar la calidad de los datos.

\begin{itemize}
  \item Una organización que no hace nada ante la mala calidad de datos está eligiendo la opción uno.
  \item La opción dos es la que se utiliza más frecuentemente. Ya que existe un gran número de herramientas informáticas que permiten
    automatizar la corrección de errores. Para luego corregir dichos errores a traves de la intuición o reglas fijas. El problema de esto
    es que es una solución a corto plazo ya que no estamos detectando la raíz del problema solo estámos corrigiendo los errores y permitimos
    que se sigan generando. Es decir la limpieza nunca termina, lo que conlleva mucho tiempo genera más gastos y da una falsa sensación de
    seguridad.
  \item Realizar limpiezas periodicamente es lo mismo que la corrección de errores solo que se realiza más frecuentemente por lo tanto
    conlleva los mismo errores anteriormente mencionados. Solo se corrigen los datos errones generados pero nunca se corrige la raíz del
    problema es decir lo que está generando esos datos.
  \item La ultima opción se basa en la prevención ya que identifico la fuente de error de esos datos. Este plan se basa en la detección y
    corrección es decir en la prevención. Esto nos provee ventajas a largo plaza pero dicho proceso de detección y corrección es más costoso
    y requiere más tiempo ya que se deben revisar los procesos de la organización y las fuentes de información para buscar una solución y
    por lo general las empresas están pensando en el día a día.
\end{itemize}


Resumen:
\begin{itemize}
  \item Una mala calidad de datos puede generar el fracaso total de la utilidad de una base de datos.
  \item Los datos son medidos y evaluados de acuerdo al uso de los mismos.
  \item La calidad de los datos es un concepto multidimensional.
  \item Hay diferentes enfoques para enfrentar la calidad de datos. Detección y corrección vs prevención.
\end{itemize}
