\documentclass{article}

% Language setting
% Replace `english' with e.g. `spanish' to change the document language
\usepackage[spanish]{babel}

% Set page size and margins
% Replace `letterpaper' with `a4paper' for UK/EU standard size
\usepackage[letterpaper,top=2cm,bottom=2cm,left=3cm,right=3cm,marginparwidth=1.75cm]{geometry}
\usepackage{titlesec}

% Useful packages
\usepackage{amsmath}
\usepackage{graphicx}
\usepackage[colorlinks=true, allcolors=blue]{hyperref}
\usepackage{tikz}
\usetikzlibrary{shapes.geometric, arrows}

% we want ER + above/below + left/right
\usetikzlibrary{er,positioning}
\tikzstyle{startstop} = [rectangle, minimum width=3cm, minimum height=1cm,text
centered, draw=black, fill=blue!15]
\tikzstyle{arrow} = [thick]

\title{Resumen de Diseño de Bases de Datos}
\author{You}

\begin{document}
\maketitle

\section{Unidad 1 - Modelo Entidad Relación}

  \subsection{Estructura del Sistema de Base de Datos}

    Un sistema gestor de base de datos consiste en una colección organizada de
    elementos de datos interrelacionados (DB) y un conjunto de programas para
    almacenar, modificar y acceder dichos elementos de datos.
    El proposito de un sistema gestor de base de datos es proporcionarle al
    usuario una visión abstracta de los datos, es decir esconder ciertos
    detalles de como se almacenan y mantienen los datos. \\
    Esto se logra a través de tres \textbf{niveles de abstracción} sobre la
    realidad física de los datos. 

    \begin{itemize}
      \item Nivel de vistas: Corresponde a grupos de usuarios o a programas de
        aplicación.
      \item Nivel lógico: Corresponde a los datos y las relaciones entre ellos.
      \item Nivel físico: Corresponde a como se almacenan los datos.
    \end{itemize}

    Algunas de las tareas que lleva a cabo un sistema gestor de base de datos
    son:

    \begin{itemize}
      \item Evita \textbf{redundancia} e \textbf{inconsistencia} de datos.
      \item Permite un acceso eficiente a los datos.
      \item Aisla los datos de las aplicaciones.
      \item Ejecuta las transacciones siguiendo el principio de
        \textbf{atomicidad}.
      \item Controla el acceso concurrente a los datos.
      \item Otorga recuperación de la base de datos en caso de fallas, errores
        de varios tipos o usos mal intencionados.
    \end{itemize}
    % \begin{enumerate}
    %   \item Crear bases de datos y sus esquemas (schemas - logical structure of the
    %     data) a través de un lenguaje de definición de datos (data-definition
    %     language).
    %   \item Agregar, quitar o modificar datos a través de un lenguaje de
    %     manipulación de datos (data-manipulation language).
    %   \item Recuperación de la base de datos en caso de fallas, errores de varios
    %     tipos o malos usos intencionados.
    %   \item Controla el accesso a los datos por parte de varios usuarios a la vez,
    %     no permite interacciones inesperadas entre los usuarios (isolation) y
    %     tampoco permite acciones parcialmente terminadas sin completar (atomicity).
    % \end{enumerate}

  \subsection{Base de Datos}
    Una base de datos contiene información de un dominio en particular. Las
    bases de datos van cambiando a medida que pasa el tiempo ya que
    constantemente se insertan, modifican o borran datos de la misma. \\ Al
    diseño completo de la base de datos se lo denomina \textbf{esquema de base
    de datos}. Si capturamos la base de datos en un momento en particular
    obtendremos una \textbf{instancia de base de datos}. Esta instancia nos
    refleja el estado de una base de datos en un momento determinado. 

  \subsection{Niveles de la Base de Datos}
  Si implementamos niveles de abstracción para un esquema de base de datos
  podemos modificar la definición del esquema sin que afecte a un nivel superior
  de abstracción. Esto nos provee:
    \begin{itemize}
      \item Independencia física de los datos.
      \item Independencia lógica de los datos.
    \end{itemize}

  \subsection{Modelo de Datos}
  Un modelo de datos es un conjunto de herramientas conceptuales que nos
  permiten describir los datos, relaciones entre los datos, semántica de los
  datos y restricciones de integridad.

  \subsection{Modelo Entidad Relación}
    \subsubsection{Entidad}
    Las entidades guardan un tipo de elementos relacionados, descritos por
    atributos.
%     \item Entidad Debil: Son entidades que tienen una clave que no es suficiente para identificar los elementos de dicha entidad. Se debe marcar el discriminante el cual debería ser clave pero necesita la clave de la entidad fuerte para poder identificar a dicha entidad. Se denota con linea punteada.

    \subsubsection{Atributo}
    Los atributos son datos atómicos que sirven para describir a los elementos que pertenecen a una entidad. Los atributos poseen un dominio
    el cual indica los valores permitidos para dicho atributo. Existen distintos tipos de atributos:
      \begin{itemize}
          \item Simple: Contienen un solo valor. Se representa con un óvalo.
          \item Compuesto: Agrupa atributos que están relacionados entre sí. Por ejemplo el atributo nombre puede estar compuesto por primer
            nombre, apodo, apellido.
          \item Multivaluado: Puede tener más de un valor. Por ejemplo, grado académico, teléfono. Se representa con doble óvalo.
          \item Derivado: Se calcula a partir de otros datos del modelo. Se representa con un óvalo punteado.
          \item Clave: Toda entidad debe tener un atributo que la identifique de manera unívoca. Puede ser simple o compuesto. Se representa
            subrayando el contenido del atributo.
      \end{itemize}
        \subsubsection{Clave}
        Las claves nos permiten identificar unívocamente una entidad. Existen
        distintos tipos de claves:
          \begin{itemize}
            \item Primaria: Es/son los atributos que identifican unívocamente a una entidad.La clave primaria cuenta con dos propiedades que
              son unicidad y irreducibilidad. Unicidad  ya que no puede haber dos tuplas con el mismo valor en la clave primaria e
              irreducibilidad quiere decir que si quito un atributo ya no me alcanza para identificar univocamente a la entidad.
            \item Superclave: Es un conjunto de atributos que tiene incluida la clave primaria pero a su vez tiene más atributos.
            \item Candidata: Es un atributo que se encuentra en un conjunto de posibles claves primarias.
            \item Foránea: Son aquellos atributos que pueden o no ser clave primaria pero son claves primarias en otra entidad/relación.
            \item Compuesta: Es una clave primaria o candidata que tiene más de un atributo.
          \end{itemize}
        
          \subsubsection{Relación} Indica como se relacionan las entidades en el modelo. Existen relaciones 1:1, 1:N, N:M. También existen
          relaciones recursivas las cuales indican una relación entre elementos de la misma entidad. Se representan marcando los dos roles
          de la relación. Se representan con un rombo. \\ En una relación una entidad puede tener participación total o parcial en dicha
          relación. Si es una participación total todo elemento de dicha entidad debe estar relacionado con las demás entidades que
          participen de la relación. Las participaciones parciales son lo opuesto a las totales. Las participaciones totales se representan
          con doble linea y las participaciones parciales con una sola linea. Si una relación posee atributos se denomina relación
          descriptiva.

          \subsubsection{Especialización} Nos permite diseñar un subconjunto de entidades dentro de otra entidad. Las especializaciones
          pueden ser solapadas o dijuntas. Por ejemplo: si tenemos una entidad persona las especializaciones de empleado y cliente nos
          permiten distinguir los distintos tipos de personas dentro de un dominio determinado. 

          \subsubsection{Generalización} Cuando tenemos atributos en común entre entidades podemos expresar una generalización. Se agrupan
          las entidades que poseen dichos atributos bajo una nueva entidad que contiene todos los atributos en común y describe
          semánticamente las entidades dentro de este grupo.

          % \subsubsection{Especialización/Generalización} Si es dijunta indica que una entidad no puede pertenecer a más de una
          % especialización se representa con una D. Si es Solapada permite que una entidad pertenezca a más de una especialización se
          % representa con una O. Otra categoría es si es parcial o dijunta. Esto nos indica si una especialización es total toda entidad debe
          % pertenecer a una de las especializaciones. En cambio si es dijunto una entidad puede o no pertenecer a dicha especialización. La
          % especialización total y dijunta es la más restrictiva de las especializaciones.

          \subsubsection{Unión} Nos permite encasillar a varias entidades distintas dentro de un rol o una categoría. Estas
          entidades no son similares semánticamente pero si comparten el mismo rol. Las uniones también pueden ser totales o parciales.

          % \item Derivación union: Se deriva de la misma forma que una relación uno a muchos

\section{Unidad 2 - Teoría y diseño de bases de datos relacionales}

  \subsection{Formas Normales}
  Una Forma Normal es una restricción sobre un esquema de Base de Datos relacionales con dependencias funcionales que previene ciertas propiedades indeseables. Cumplir ciertas formas normales garantiza buenos diseños de bases de datos.
  El propósito de las formas normales es evitar redundancia y eliminar inconsistencias y anomalías de inserción y borrado.

\subsection{Dependencias Funcionales}
Una dependencia funcional es una restricción de valor único que se aplica sobre dos conjuntos de atributos X e Y de un esquema de relación de una base de datos. Constituye entonces una generalización del concepto de clave. Nos permite expresar hechos sobre el dominio que estamos modelando con nuestra base de datos. Esta restricción especifica que el valor del o los atributos en X determinan un único valor de los atributos en Y en todos los estados de la relación. Estás dependencias las utilizamos para normalizar los esquemas de relaciones de una base de datos.

\subsection{Primera forma normal - 1FN}
Un esquema de relación R está en 1FN si los valores del dominio de A son atómicos, siendo A cada uno de los atributos A en R, es decir las valores de los atributos no son listas, cojuntos de valores o valores compuestos.

La 1FN establece que los dominios de los atributos deben incluir solo valores atómicos (simples, indivisibles) y prohíbe entonces las “relaciones dentro de relaciones” o las “relaciones como atributos de tuplas”

Ventajas de la 1FN

\begin{itemize}
    \item Representación Tabular
    \item Lenguajes de Consulta más simples
    \item Definición de restricciones más simples
\end{itemize}

Desventajas de la 1FN

\begin{itemize}
    \item Inconsistencia.
    \item Redundancia.
    \item Anomalías de inserción y de borrado.
\end{itemize}
Ejemplo de un esquema de relación en 1FN: \\
\begin{table}[h]
\begin{tabular}{| l | l | l | l | l| l|}
\hline
Id\_empleado & Dni & Nombre & Apellido & Id\_proyecto & Descripción\_proyecto\\ \hline
1 & 45789777 & Martin & Perez & A1 & Proyecto sobre microprocesadores\\ \hline
2 & 45789777 & Lucas & Martinez & A1 & Proyecto sobre programación\\ \hline
3 & 33765922 & María & García & A2 & Proyecto sobre compiladores\\ \hline
4 & 33765922 & Alejandra & Carrizo & A2 & Proyecto sobre compiladores\\ \hline
5 & 33444222 & Berta & Gonzalez & A3 & Proyecto sobre compiladores\\ \hline
\end{tabular}
\end{table}

\begin{itemize}
    \item Inconsistencia: En este ejemplo podemos ver inconsistencia en los datos ya que en la tupla con id\_empleado 1 y 2 podemos ver el mismo proyecto pero con distintas descripciones.
    \item Redundancia: Si vemos las tuplas  3 y 4 vemos que se repite información en este caso la descripción del proyecto.
    \item Anomalía de Inserción: Si insertamos una tupla con id\_proyecto A2 con descripción distinta a la de las tuplas ya existentes no podemos saber cual es la descripción correcta del proyecto A2.
    \item Anomalía de Borrado: Si borramos la tupla con id\_proyecto a3 perdemos la información tanto del proyecto como de la persona.
\end{itemize}

\subsection{Segunda forma normal - 2FN}
Un esquema de relación R está en 2FN si está en 1FN y para cada atributo no primo en R tiene una dependencia funcional total con alguna clave de R.

Es decir cada atributo no primo es irreduciblemente dependiente de una clave primaria.


\begin{table}[h]
    \centering
    \begin{tabular}{cccc}
        Torneo & Año & Ganador & Fecha de nacimiento del ganador\\
        Des Moines Masters & 1998 & Chip Masterson & 14 de marzo de 1977\\
        Indiana Invitational & 1998 & Al Fredickson & 21 de julio de 1975\\
        Cleveland Open & 1999 & Bob Albertson & 28 de septiembre de 1968\\
        Des Moines Masters & 1999 & Al Fredrickson & 21 de julio de 1975\\
        Indiana Invitational & 1999 & Chip Masterson & 14 de marzo de 1977\\
    \end{tabular}
    \caption{Caption}
    \label{tab:my_label}
\end{table}

\vspace{1cm}
Esta relación está en 2FN pero presenta anomalía de actualización por ejemplo modificar la fecha de nacimiento provocaría que el mismo ganador tuviera dos fechas de nacimiento distintas.

\subsection{Tercera forma normal - 3FN}
Un esquema de relación R está en 3FN con respecto a un conjunto de dependencias funcionales F si está en 2FN y cada atributo no primo no es transitivamente dependiente de una clave.

Segunda definición
Un esquema de relación R está en 3FN con respecto a un conjunto de dependencias funcionales F si para cada dependencia funcional X -> A en F con A no perteneciente a X se verifica que: 
\begin{itemize}
    \item X es superclave o
    \item A es primo
\end{itemize}

\section{Unidad 3 - Modelo Relacional}

\subsection{Dependencias Funcionales}
Existe una teoría de diseño para las relaciones que nos permite examinar un diseño cuidadosamente y realizar mejoras basándonos un unos pocos principios.
La teoría comienza haciéndonos establecer las restricciones que se aplican a la relación. La restricción más común es la dependencia funcional es una declaración de un tipo que generaliza la idea de una clave para la relación.
En la teoria del diseño de bases de datos relacionales existe un concepto llamado dependencias funcionales el cual es una declaración de tipo que generaliza la idea de una clave para una relación.
Una Dependencia funcional sobre una relación R es una proposición de la forma:
Si dos tuplas de r coinciden en las atributos A1, A2, ... , An entonces también coincidirán en otro atributo B.
A1, A2, ... , An => B es decir decimos que A1, A2, ... An determinan funcionalmente a B
Es decir las tuplas poseen los mismos valores en sus componentes respectivas para cada uno de esos atributos entonces también coincidirán en otro atributo B.

\subsection{Restricciones de Integridad Referencial}
Antes de definir que es una restricción de integridad referencial debemos definir el concepto de clave primaria y clave foránea.
Una clave primaria es un atributo el cual nos permite identificar unívocamente a una entidad y por lo tanto identificar unívocamente a cada tupla de esa entidad. Ahora podemos definir que es una clave foránea. Una clave foránea es un atributo que es clave primaria de otra relación.
Las restricciones de integridad referencial son especificadas entre dos relaciones y se utilizan para mantener la consistencia entre las tuplas en ambas relaciones. Es decir una tupla de una relación A que hace referencia a una tupla de otra relación B debe referir a una tupla existente de esa relación B.
Con esto logramos que nuestra base de datos se mantenga en un estado consistente.

Sobre la clave foránea se definen un conjunto de reglas que son:
\begin{itemize}
\item Regla de nulos - inserción?: Establece si es posible colocar nulos en las claves foráneas.
\item Regla de Borrado: Establece restricciones al momento de eliminar una tupla en la base de datos.
\item Regla de actualización: Establece restricciones al momento de actualizar una tupla de la base de datos.
\end{itemize}
Reglas de actualización: Si se intenta actualizar el id de una tupla y a su vez dicho id es clave foranea en otra relación de la base de datos se aplican las siguientes reglas: 
\begin{itemize}
\item Restringir: Cuando se actualiza una clave primaria en una relación y dicha clave es clave foranea en otras relaciónes esta regla evita dicha modificación.
\item Propagar: Cuando se actualiza la clave primaria de una tupla y dicha clave es clave foranea en otras relaciones esta regla actualiza la clave foranea en todas las tuplas que la utilicen.
\item Colocar Nulos: Al actualizar el id de una tupla en una base de datos si dicho id es clave foranea en otra relación esta regla setea dichas claves en null.
\end{itemize}
Reglas de Borrado: Si se intenta borrar una tupla de la cual su id es clave foranea en alguna otra relación en la base de datos. Las reglas actuan de la siguiente manera:
\begin{itemize}
    \item Restringir: Esta regla restringe ese comportamiento y no lo permite.
    \item Propagar: Esta reglar borra la tupla seleccionada y luego borra todas las tuplas en la base de datos que tenga como clave foranea el id de la tupla borrada.
    \item Colocar nulos: Al borrar una tupla en una entidad esta regla reemplaza el id de la tupla eliminada en las relaciones que tienen dicho id como clave foranea por el valor null.
\end{itemize}

\subsection{Algebra relacional y calculo de dominios}
El algebra relacional es el conjunto basico de operaciones para el modelo relacional. Estas operaciones permiten al usuario construir consultas como expresiones de algebra relacional. El algebra relacional es muy importante por varias razones.
\begin{itemize}
    \item Provee los fundamentos formales prar las operaciones del modelo relacional.
    \item Se utiliza como base para la implementación y optimización de consultas en el proceso de consulta y son una parte integrl en los modulos de optimización en las los sistemas gestores de bases de datos relacionales.
    \item Algunos de sus conceptos son incorporados dentro de lenguaje de consulta SQL standard.
\end{itemize}
El cálculo relacional provee un lenguaje declarativo de alto nivel para especificar consultas relacionales.
\vspace{2em}
\begin{center}
   \begin{tikzpicture}[node distance=2cm]
        \node (empleado) [startstop] {Empleado};
        \node (departamento) [startstop, below of= empleado] {Departamento};
        \draw [arrow] (empleado) -- (departamento);
    \end{tikzpicture} 
\end{center}

\end{document}
